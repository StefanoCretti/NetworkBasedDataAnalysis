Data analysis is a complex multi step process. Formulate the question.

\section{T-test}
T-test returns a \textit{p}-value,  ... %TODO (aggiungi le note)

The T-test produces more reliable results with normal distributed data.

The t-test suffer of the multiplicity problem. A solution is touse a correction method, like Bonferroni. The problem of threshold is not already solved, normally 0.05\%. The percentage remains arbitrary. 0.05 means that you have a mistake as result only for the 5\% of the times. 
The ones with smaller \textit{p}-value are the most interesting. It is possible to sort the genes and take those with lower values of \textit{p}-value.


\begin{itemize}
	\item Check data in there 
	\item Not too many NAs
	\item Inspect
	\item Normalization: The sistematic source of variation can be eliminated by using a box-plot, which means great amount of variation. normalization is made by subtracting the median.
Align the size of the boxed by aligning the standard deviation
\end{itemize}


\section{Data transformation}
It has not to be performed without a valid reason
The simplest is the log transformation.
Rank transformation 

The z-score is also a transformation, 

\section{PCA} %TODO
The result of PCA is normally a 2-dimensional plot, PCA is used in a number of fields. X coordinates to y coordinates. 
For the second direction it has to be selected  a component which is perpendicular. When a cloud. Once identified the set of coefficients.

Each eigenvector fills a row of the matrix. The eigenvalues provide an estmate of the percentage of the variability which 

hOW MANY COMPONENTS TO USE
the first few components have to be used. 

Y are new coordinates, 








Covariance matrix vs correlation matrix: 



